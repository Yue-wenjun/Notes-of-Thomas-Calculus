\documentclass{book}
\usepackage{newtxtext}
\usepackage[marginparwidth={4cm},lmargin={1cm}, rmargin={5cm}]{geometry}
\usepackage[dvipsnames,svgnames,x11names]{xcolor}
\usepackage[strict]{changepage} % 提供一个 adjustwidth 环境
\usepackage{framed} % 框包
\usepackage{xeCJK}%中文包
\usepackage{amsmath}%写连等式对齐包
\usepackage{graphicx}%插入图片
\usepackage{wallpaper}%插入背景图
\usepackage{indentfirst}%latex第一段首行不缩进,看着不爽,于是用包使每段都缩进
%\usepackage{titlesec}如有需要设置各级标题
%\usepackage{ctex}如有需要设置中文字体


\geometry{a4paper,centering,scale=0.8}
\definecolor{formalshade}{rgb}{0.95,0.95,1} % 文本框颜色
% ------------------******-------------------
% 注意行末需要把空格注释掉,不然画出来的方框会有空白竖线
\definecolor{greenshade}{rgb}{0.92,1,0.92}
\definecolor{grayshade}{rgb}{0.90,0.90,0.90}
%蓝紫框--------------------------------------------
\newenvironment{formal1}{%
\def\FrameCommand{%
\hspace{1pt}%
{\color{DarkBlue}\vrule width 2pt}%
{\color{SteelBlue}\vrule width 4pt}%
\colorbox{formalshade}%
}%
\MakeFramed{\advance\hsize-\width\FrameRestore}%
\noindent\hspace{-4.55pt}% disable indenting first paragraph
\begin{adjustwidth}{}{7pt}%
\vspace{2pt}\vspace{2pt}%
}
{%
\vspace{2pt}\end{adjustwidth}\endMakeFramed%
}
%绿框---------------------------------------
\newenvironment{formal2}{%
\def\FrameCommand{%
\hspace{1pt}%
{\color{Green}\vrule width 2pt}%
{\color{DarkSeaGreen1}\vrule width 4pt}%
\colorbox{greenshade}%
}%
\MakeFramed{\advance\hsize-\width\FrameRestore}%
\noindent\hspace{-4.55pt}% disable indenting first paragraph
\begin{adjustwidth}{}{7pt}%
\vspace{2pt}\vspace{2pt}%
}
{%
\vspace{2pt}\end{adjustwidth}\endMakeFramed%
}
%灰框---------------------------------------
\newenvironment{formal3}{%
\def\FrameCommand{%
\hspace{1pt}%
{\color{darkgray}\vrule width 2pt}%
{\color{Ivory4}\vrule width 4pt}%
\colorbox{grayshade}%
}%
\MakeFramed{\advance\hsize-\width\FrameRestore}%
\noindent\hspace{-4.55pt}% disable indenting first paragraph
\begin{adjustwidth}{}{7pt}%
\vspace{2pt}\vspace{2pt}%
}
{%
\vspace{2pt}\end{adjustwidth}\endMakeFramed%
}
%天蓝框---------------------------------------
\newenvironment{formal4}{%
\def\FrameCommand{%
\hspace{1pt}%
{\color{SkyBlue}\vrule width 2pt}%
{\color{Cyan1}\vrule width 4pt}%
\colorbox{LightCyan1}%
}%
\MakeFramed{\advance\hsize-\width\FrameRestore}%
\noindent\hspace{-4.55pt}% disable indenting first paragraph
\begin{adjustwidth}{}{7pt}%
\vspace{2pt}\vspace{2pt}%
}
{%
\vspace{2pt}\end{adjustwidth}\endMakeFramed%
}
%粉框---------------------------------------
\newenvironment{formal5}{%
\def\FrameCommand{%
\hspace{1pt}%
{\color{HotPink}\vrule width 2pt}%
{\color{LightPink}\vrule width 4pt}%
\colorbox{MistyRose}%
}%
\MakeFramed{\advance\hsize-\width\FrameRestore}%
\noindent\hspace{-4.55pt}% disable indenting first paragraph
\begin{adjustwidth}{}{7pt}%
\vspace{2pt}\vspace{2pt}%
}
{%
\vspace{2pt}\end{adjustwidth}\endMakeFramed%
}
%首段首行缩进--------------------------------------
\setlength{\parindent}{2em}

% ------------------******-------------------
% ------------------******-------------------
% ------------------******-------------------
% ------------------******------------------
\begin{document}
 \begin{titlepage}
 \newgeometry{left=3cm,right=3cm}
 \ThisCenterWallPaper{1.28}{2.jpg}
 \color{darkgray} 
 \begingroup
 \thispagestyle{empty}
 \centering
 \vspace*{5cm}
 \par\normalfont\fontsize{35}{35}\sffamily\selectfont
 \textbf{Thomas Calculus}\\ 
 {\LARGE 002 Calculus and Technical English}\par % Book title
 \vspace*{1cm}
 {\Huge Book Notes}\par % Author name
 \endgroup
 \end{titlepage}



 \section{Functions, Graphs, and Lines}
\marginpar{parallel to 平行\\
vertical line  垂线\\
inverse 反函数\\
Interval notation 区间表示法\\
Polynomials\\
wiggle 摆动\\
Rational functions\\
Exponentials and logarithms\\
Trig functions\\
functions involving absolute values}


\par \begin{enumerate}
 \item functions: their domain, codomain, and range, and the vertical line test;
 \item inverse functions and the horizontal line test;
 \item composition of functions;
 \item odd and even functions;
 \item graphs of linear functions and polynomials in general, as well as a brief
 survey of graphs of rational functions, exponentials, and logarithms; and
 \item how to deal with absolute values.

\end{enumerate}
 \color{darkgray}
 \definecolor{shadecolor}{rgb}{0.90,0.9,0.90}
 \begin{shaded}
 {\subsection[short]{function}}
\end{shaded}
\color{black}

\begin{formal4}
about inverse\\    
If the domain of a function $f$ can be restricted so that $f$ has an inverse
$f^-1$then
\begin{itemize}
\item $f (f ^{−1}(y)) = y$ for all $y$ in the range of $f$ ; but
\item $f ^{−1}(f (x))$ may not equal $x$; in fact, $f ^{−1}(f (x)) = x$ only when x is in
the restricted domain.\\
\end{itemize}
about composition notation\\
Another way of expressing $f (x) = h(g(x))$ is
to write $f = h \circ g$; here the circle means “composed with.” That is, $f$ is $h$
composed with $g$, or in other words, $f$ is the composition of h and g. What’s
tricky is that you write $h$ before $g$ (reading from left to right as usual!) but
you apply $g$ first.
\end{formal4}

\section{Review of Trigonometry}
\begin{itemize}
    \item angles in radians and the basics of the trig functions;
    \item trig functions on the real line (not just angles between 0◦ and 90◦);    
    \item graphs of trig functions; and
    \item trig identities.
\end{itemize}

\section{Introduction to Limits}
\marginpar{intuitive 直观的\\
}
\begin{itemize}
    \item an intuitive idea of what a limit is;
    \item left-hand, right-hand, and two-sided limits, and limits at $\infty$ and $-\infty$;
    \item when limits fail to exist; and
    \item the sandwich principle (also known as the “squeeze principle”).
\end{itemize}
\begin{formal5}

But here’s something important: the regular two-sided limit at x = a exists exactly
when both left-hand and right-hand limits at x = a exist and are equal to
each other!

we now have a formal definition of the
term “vertical asymptote”:
“f has a vertical asymptote at x = a” means that at least one
of limx → a+f (x) and limx → a−f (x) is equal to ∞ or −∞.

\end{formal5}

Implicent Differentiation


\end{document}